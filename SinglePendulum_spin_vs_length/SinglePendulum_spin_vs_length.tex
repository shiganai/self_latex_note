\documentclass[a4paper,11pt]{jsarticle}


% 数式
\usepackage{amsmath,amsfonts,amssymb}
\usepackage{bm}
% 画像
\usepackage[dvipdfmx]{graphicx}
\usepackage[dvipdfmx]{color}
\usepackage{siunitx}
\usepackage{wrapfig}
\usepackage{cases}
\usepackage{dcolumn}
\usepackage{subcaption}


% add hyperlinks
\usepackage[dvipdfmx]{hyperref}
\usepackage{pxjahyper}
\hypersetup{
  colorlinks=true,
  linkcolor=blue,
  citecolor=blue,
  breaklinks=true, 
}

% next page in align
\allowdisplaybreaks[1]

\begin{document}

\title{単振り子の長さと回転数の関係について}
\author{Hiro Hirabayashi}
\date{\today}
\maketitle

\begin{figure}[h]
  \begin{tabular}{cc}
    \begin{minipage}[t]{0.45\textwidth}
      \centering
      \includegraphics[width=1\textwidth]{config.png}
      \caption{設定}
      \label{config.png}
    \end{minipage} &
    \begin{minipage}[t]{0.45\textwidth}
      \centering
      \includegraphics[width=1\textwidth]{length_after_takeoff.png}
      \caption{離軸後の慣性モーメント}
      \label{length_after_takeoff.png}
    \end{minipage}
  \end{tabular}
\end{figure}

図\ref{config.png}のような状況で、剛体振り子が回転軸から離れる状況を考える。
ただし、回転軸から離れた後は図\ref{length_after_takeoff.png}のように、
離軸前の振り子の長さ($\ell^L, \ell^S$)によらず、$\ell^{Air}$に変化する。
これは平行棒の後方宙返りにおいて、離手後に同じ姿勢をとる状況を再現するためである。

今回は、質量が一定に保たれ、密度が一様な剛体振り子を考えるが、
力学的エネルギーを長さ$\ell$によらず一定に保つために、
水平$\theta = 0$から振り出すことを考える。

\section{"単振り子は長いほうが高い回転数を実現する"の証明}

単振り子の良い点は、時間パラメータ$t$に対して解析的な解を求めなくても、
角度パラメータ$\theta$に対して簡単に解析的な解が求められることである。
2重振り子にはないこの特性は、
簡単に立てられるエネルギー保存則に含まれる変数が$\theta$とその微分である$\omega$
だけであることによる。

エネルギーの保存則より
\begin{align*}
  0 
  &= \frac{1}{2} m \Big\{ (\ell\omega\sin\theta)^2 + (-\ell\omega\cos\theta)^2  \Big\}
  + \frac{1}{2} \cdot \frac{1}{3} m\ell^2 \cdot \omega^2
  + mg (-\ell \sin\theta )
  \\
  &= \frac{1}{2}m\ell^2\omega^2 + \frac{1}{6}m\ell^2\omega^2 + mg (-\ell\sin\theta)
  \\
  &= \frac{2}{3}m\ell^2\omega^2 + mg (-\ell\sin\theta)
\end{align*}
これにより
\begin{align*}
  \omega^2 &= \frac{3}{2m\ell^2}mg\ell\sin\theta = \frac{3g}{2\ell}\sin\theta,
  \\
  \therefore
  \omega &= \sqrt{ \frac{3g}{2\ell}\sin\theta }
\end{align*}
が求まる。
空中での回転速度を$\Omega$とすると
\begin{gather}
  I\Omega 
  = \frac{1}{3}m\ell^2\omega
  = \frac{1}{3}m\ell^2 \sqrt{ \frac{3g}{2\ell}\sin\theta }
  = \sqrt{ \frac{m^2 g}{6} \ell^3 \sin\theta }
  \notag
  \\
  \therefore \Omega
  = \frac{1}{I} \sqrt{ \frac{m^2 g}{6} \ell^3 \sin\theta }
  \label{eq:Omega}
\end{gather}

ところで、回転数$N_r$は滞空時間$T_{air}$、空中での回転速度$\Omega$を用いて
\begin{align*}
  N_r = \Big( \theta + T_{air} \cdot \Omega \Big) / 2\pi
\end{align*}
で表される。
"単振り子は長いほうが高い回転数を実現する"
ことを証明するには
例えば$\theta, \Omega$が同じ状況で、
$T_{air}$が長さに対して単調増加することを示せばよいが、
それは証明できない。
そもそも$\theta, \Omega$が同じ状況が用意できないが、
例えば$\theta=\frac{1}{2}\pi$での離軸を考えると良い。
振り子が短いほうが$T_{air}$は大きいが、
振り子が長いほうが$\Omega$は大きかったりする。
しかし、
離軸後の最高到達点の高さが一定である場合、
$\theta, T_{air}, \Omega$が
長さ$\ell$に対して単調増加することが示せる。

離軸後の最高到達点の高さが一定であるような条件は
鉛直方向のエネルギー保存を考えると簡単に表される。
離軸後の最高到達点の高さを$h$とする。
振り子は力学的エネルギーがすべて位置エネルギーに変換された場合の高さ$0$を超えることはないので、
$h\leq0$が成り立つ。
\begin{align}
  mgh 
  &= \frac{1}{2}m (-\ell \omega \cos \theta)^2 + mg (-\ell \sin \theta)
  \label{eq:energy_vertical}
  \\
  &= \frac{1}{2}m\ell^2 \omega^2 \cos^2 \theta + mg (-\ell \sin \theta)
  \notag
  \\
  & \ \ \left( \because \omega^2 = \frac{3g}{2\ell}\sin\theta \right) \notag
  \\
  &= \frac{1}{2}m\ell^2 \cos^2 \theta \cdot \frac{3g}{2\ell}\sin\theta + mg (-\ell \sin \theta)
  \notag
  \\
  &= \frac{3mg\ell}{4}\sin\theta \cos^2\theta + mg (-\ell\sin\theta)
  \notag
  \\
  &= mg\ell\cos\theta \left( \frac{3}{4}\cos^2\theta - 1\right)
  \notag
  \\
  & \ \ \left( \because \cos^2\theta = 1 - \sin^2\theta \right)
  \notag
  \\
  &= mg\ell\sin\theta \left( \frac{3}{4} - \frac{3}{4}\sin^2\theta - 1 \right)
  \notag
  \\
  &= mg\ell\sin\theta \left( - \frac{3}{4}\sin^2\theta - \frac{1}{4} \right)
  \notag
  \\
  &= -\frac{1}{4}mg\ell\sin\theta (3\sin^2\theta + 1)
  \notag
\end{align}
よって長さ$\ell$と離軸角度に関して以下の式が得られる。
\begin{align}
  mgh &= -\frac{1}{4}mg\ell \sin\theta (3\sin^2\theta + 1)
  \label{eq:energy_vertical:conclusion}
  \\
  \Leftrightarrow
  \ell &= \frac{4(-h)}{\sin\theta (3\sin^2\theta + 1)} \ \ \ \ \Big( > 0, \ \ \because -h > 0, \sin\theta > 0 \Big)
  \label{eq:l_vs_sin}
\end{align}

平行棒の後方宙返りを模すために$\frac{1}{2}\pi \leq \theta \leq \pi$で離軸することを考える。
この区間で$\sin\theta$は単調減少するから
$\sin\theta (3\sin^2\theta)$ も単調減少する。
長い剛体振り子に対応する変数Xを$X^L$、
短い剛体振り子に対応する変数Xを$X^S$と表すと、
\begin{align*}
  \ell^S < \ell^L
\end{align*}
であるから、
式\ref{eq:l_vs_sin}によって
\begin{align*}
  \frac{1}{\sin\theta^S (3\sin^2\theta^S + 1)} &< \frac{1}{\sin\theta^L (3\sin^2\theta^L + 1)}
  \\
  \Leftrightarrow
  \sin\theta^S (3\sin^2\theta^S + 1) &> \sin\theta^L (3\sin^2\theta^L + 1)
  \\
  \Leftrightarrow
  \sin\theta^S &> \sin\theta^L
  \\
  \Leftrightarrow
  \theta^S &< \theta^L
\end{align*}
が分かる。
離軸後の最高到達点の高さが一定であるような条件において
長さ$\ell$に対して離軸時の角度$\theta$が単調増加であることが示された。

また、滞空時間$T_{air}$は
\begin{align*}
  T_{air} = (\textrm{離軸から最高高度到達まで}) + (\textrm{最高高度到達以降})
\end{align*}
で表され、
最高高度到達以降は"離軸後の最高到達点の高さが一定"によって一致するから、
$T_{air}$の差異は離軸から最高高度到達までにかかる時間によって生じる。
離軸時の鉛直方向速度$v_y$は
\begin{align*}
  v_y = \ell\omega (-\cos\theta) \ \ \ \ \Big( > 0, \ \ \because \cos\theta < 0 \Big)
\end{align*}
によって現れるが、
離軸後の最高到達点の高さが一定であるような条件において
長さ$\ell$に対して$T_{air}$が単調増加することを証明するのに
特に$\theta, \omega$で表す必要はない。
鉛直方向に関するエネルギー保存則の式\ref{eq:energy_vertical}を$v_y$で書き換えると
\begin{align*}
  mgh = \frac{1}{2}m{v_y}^2 + mg (-\ell \sin\theta)
\end{align*}
\begin{align*}
  \Leftrightarrow
  \frac{1}{2}m{v_y}^2
  &= mgh + mg\ell\sin\theta
  = mgh + mg\sin\theta\cdot\frac{4(-h)}{\sin\theta(3\sin^2\theta + 1)}
  \\
  &= mg\left\{ h + \frac{4(-h)}{3\sin^2\theta + 1} \right\}
\end{align*}
\begin{align*}
  {v_y}^2 = 2g(-h)\left\{ \frac{4}{3\sin^2\theta + 1} - 1 \right\}
\end{align*}
よって
\begin{align*}
  \ell^S &< \ell^L
  \\
  \Leftrightarrow
  \sin\theta^S &> \sin\theta^L
  \\
  \Leftrightarrow
  3\sin^2\theta^S + 1 &> 3\sin^2\theta^L + 1
  \\
  \Leftrightarrow
  \frac{1}{3\sin^2\theta^S + 1} &< \frac{1}{3\sin^2\theta^L + 1}
  \\
  \Leftrightarrow
  {v_y^S}^2 &< {v_y^L}^2
  \\
  \Leftrightarrow
  v_y^S &< v_y^L
  \\
  \Leftrightarrow
  T_{air}^S &< T_{air}^L.
\end{align*}
よって
離軸後の最高到達点の高さが一定であるような条件において
長さ$\ell$に対して滞空時間$T_{air}$が単調増加であることが示された。

離軸後の最高到達点の高さが一定であるような条件において
長さ$\ell$に対して離軸時の角度$\theta$と滞空時間$T_{air}$が単調増加であることから、
図\ref{same_max_height.png}が描ける。
長い振り子のほうが、位置エネルギーをうまく活用している様子が見て取れる。

\begin{figure}[h]
  \centering
  \includegraphics[width = 0.8\textwidth]{same_max_height.png}
  \caption{同じ最高到達点の高度で、長さが異なる振り子の離軸時の様子}
  \label{same_max_height.png}
\end{figure}

最後に空中での回転速度$\Omega$について考える。
$\Omega$は式\ref{eq:Omega}により
\begin{align*}
  \Omega^2 
  &= \frac{m^2g}{6I^2}\ell^3\sin\theta
  \\
  &= \frac{m^2g}{6I^2}\sin\theta \cdot \frac{64(-h^3)}{\sin^3\theta(3\sin^2\theta+1)^3}
  \\
  &= \frac{m^2g}{6I^2}\cdot \frac{64(-h^3)}{\sin^2\theta(3\sin^2\theta+1)^3}
  \\
  &= \frac{32m^2g (-h^3)}{3I^2}\cdot \frac{1}{\sin^2\theta(3\sin^2\theta+1)^3}
\end{align*}
と表されるから、
\begin{align*}
  \ell^S &< \ell^L
  \\
  \Leftrightarrow
  \sin\theta^S &> \sin\theta^L
  \\
  \Leftrightarrow
  \sin^2\theta^S(3\sin^2\theta^S + 1) &> \sin^2\theta^L(3\sin^2\theta^L + 1)
  \\
  \Leftrightarrow
  \frac{1}{\sin^2\theta^S(3\sin^2\theta^S + 1)} &< \frac{1}{\sin^2\theta^L(3\sin^2\theta^L + 1)}
  \\
  \Leftrightarrow
  {\Omega^S}^2 &< {\Omega^L}^2
  \\
  \Leftrightarrow
  \Omega^S &< \Omega^L
\end{align*}
よって
離軸後の最高到達点の高さが一定であるような条件において
長さ$\ell$に対して空中での回転速度$\Omega$が単調増加であることが示された。

これによって
離軸後の最高到達点の高さが一定であるような条件において
長さ$\ell$に対して
離軸時の角度$\theta$、
滞空時間$T_{air}$、
空中での回転速度$\Omega$
が単調増加であることが示された。

そして離軸後の最高到達点の高さ$h$は
最も最下点での離軸した場合の$-\ell$、
水平で離軸した場合の$0$の範囲内であるから、
\begin{align*}
  -\ell \leq h \leq 0
\end{align*}
であり、この範囲内の全ての実数値を実現できる。
そしてこの範囲は長さ$\ell$が大きいほど大きく、
\begin{align*}
  (\ell^S\textrm{の実現する範囲}) \subset (\ell^L\textrm{の実現する範囲})
\end{align*}
が分かる。
よって如何なるタイミングで長さ$\ell^S$の振り子が離軸しても、
その最高到達点の高さと
同じ最高到達点を持つ長さ$\ell^L$の振り子の離軸タイミングが存在し、
その離軸タイミングでの
長さ$\ell^L$の振り子の回転数は
長さ$\ell^S$の振り子の回転数より大きい。
よって長さ$\ell^S$の振り子の如何なる回転数に対しても
それより大きい回転数が長さ$\ell^L$によって実現されることが分かるから、
"単振り子は長いほうが高い回転数を実現する"ことが証明される。


適当に着地点の高さを決め、滞空時間が完全に決定される状況にすると、
離軸時の角度$\theta$、
滞空時間$T_{air}$、
空中での回転速度$\Omega$は
図\ref{result.png}のようになる。
\begin{figure}[h]
  \centering
  \includegraphics[width = 0.6\textwidth]{result.png}
  \caption{
    実際の
    離軸時の角度$\theta$、
    滞空時間$T_{air}$、
    空中での回転速度$\Omega$
    の様子
    }
  \label{result.png}
\end{figure}

\section{各長さに対して、最高回転数を実現する離軸時の特徴}

今までの考察によって
各長さに対して最高回転数を与える離軸時の角度は分からないので、
このセクションで考える。

着地点の高さを$-h_0 (h_0>0)$とする。
この$h_0$は落差ともとれる。
離軸後の最高高度$h$は式\ref{eq:energy_vertical:conclusion}により
\begin{align*}
  h = -\frac{1}{4}\ell\sin\theta (3\sin^2\theta+1)
\end{align*}
で表される。
これにより、最高高度到達後から着地点までにかかる落下時間$t_{down}$は
\begin{align*}
  \frac{1}{2}g{t_{down}}^2
  &= h - (-h_0)
  \\
  t_{down}
  &= \sqrt{ \frac{2}{g} (h + h_0)}
  = \sqrt{ \frac{2}{g} \left( -\frac{1}{4}\ell\sin\theta (3\sin^2\theta+1) + h_0 \right) }
  \\
  & = \sqrt{ \frac{2}{g} \left( h_0 - \frac{1}{4}\ell\sin\theta (3\sin^2\theta+1) \right) }
\end{align*}
離軸後に最高高度に達するまでの時間$t_{up}$は
\begin{align*}
  t_{up} 
  &= \frac{v_y}{g} = \frac{\ell\omega(-\cos\theta)}{g}
  \\
  &= \frac{\ell(-\cos\theta)}{g} \sqrt{ \frac{3g}{2\ell} \sin\theta }
  \\
  &= \sqrt{ \frac{\ell^2(-\cos\theta)^2}{g^2} \frac{3g}{2\ell} \sin\theta }
  \\
  &= \sqrt{ \frac{3}{2g} \ell\sin\theta (1-\sin^2\theta) }
\end{align*}
で表される。
よって滞空時間$T_{air}$は
\begin{align*}
  T_{air} 
  &= t_{down} + t_{up}
  \\
  &= \sqrt{ \frac{2}{g} \left( h_0 - \frac{1}{4}\ell\sin\theta (3\sin^2\theta+1) \right) }
  + \sqrt{ \frac{3}{2g} \ell\sin\theta (1-\sin^2\theta) }
\end{align*}

また、空中での回転速度$\Omega$は式\label{eq:Omega}によってすでに表されているから、
回転数$N_r$は
\begin{align}
  N_r 
  &= \theta + T_{air} \cdot \Omega
  \notag
  \\
  &= \theta + 
  \left( \sqrt{ \frac{2}{g} \left( h_0 - \frac{1}{4}\ell\sin\theta (3\sin^2\theta+1) \right) }
    + \sqrt{ \frac{3}{2g} \ell\sin\theta (1-\sin^2\theta) } 
  \right)
  \cdot
  \frac{1}{I} \sqrt{ \frac{m^2 g}{6} \ell^3 \sin\theta }
  \label{eq:spin_number_conclusion}
\end{align}
と長さ$\ell$と離軸時の角度$\theta$をパラメータとして
解析的に求められる。

しかし、長さ$\ell$が与えられたとき、
回転数が最大になる離軸時の角度$\theta$を
この式から解析的に決定することは
容易ではない。

なので
シミュレーション値だけを図\ref{Tair_Omega_at_maxspin.png}で紹介する。
長さがある程度長い部分に関して滞空時間があまり増大せず、
回転速度が大きくなる様子がみられるが、
この様子は経験的なものゆえ、この様子がどのような条件でみられるかは定かではない。

\begin{figure}[h]
  \centering
  \includegraphics[width = 0.8\textwidth]{Tair_Omega_at_maxspin.png}
  \caption{
    最高回転数を与える離軸時の角度で離軸した場合の
    滞空時間$T_{air}$と
    空中での回転速度$\Omega$
  }
  \label{Tair_Omega_at_maxspin.png}
\end{figure}

長さが短い部分に関して滞空時間が増大することは
式\ref{eq:spin_number_conclusion}から
想像できる。
最高高度到達以降の時間である$t_{down}$が
長さ$\ell$が小さいために離軸時の角度$\theta$によらず、
落差$h_0$の値によって大体定まり、
最高高度に到達するまでの時間$t_{up}$
のみが長さ$\ell$に対して増加するからである。

この点から考えると、
長さがある程度長い部分に関して滞空時間があまり増大しないことは
最高高度到達以降の時間である$t_{down}$の減少分と
最高高度に到達するまでの時間$t_{up}$の増加分が
丁度打ち消しあうような関係になっているからと説明できるかもしれない。

ちなみに、
長さがある程度長い部分に関して滞空時間があまり増大せず、
回転速度が大きくなる様子は
平行棒モデルでも観測された
(図\ref{Tari_Omega_PB.png})
。

\begin{figure}[h]
  \centering
  \includegraphics[width = 0.8\textwidth]{Tari_Omega_PB.png}
  \caption{
    平行棒モデルでの
    滞空時間$T_{air}$と
    空中での回転速度$\Omega$
  }
  \label{Tari_Omega_PB.png}
\end{figure}

\end{document}