\documentclass[a4paper,11pt]{jsarticle}


% 数式
\usepackage{amsmath,amsfonts,amssymb}
\usepackage{bm}
% 画像
\usepackage[dvipdfmx]{graphicx}
\usepackage{siunitx}
\usepackage{wrapfig}
\usepackage{cases}
\makeatletter
\newcommand{\figcaption}[1]{\def\@captype{figure}\caption{#1}}
\newcommand{\tblcaption}[1]{\def\@captype{table}\caption{#1}}
\makeatother


\begin{document}

\title{棒上宙の角度について}
\author{平林広}
\date{\today}
\maketitle

万歳姿勢(Handsup)で回りすぎず、伸身姿勢(Stretched)で回転不足にならないことが必要。
そのうえで、その中間が最も操作しやすいはず(根拠なし)。

それぞれの姿勢での回転速度$\omega_{H},\omega_{S}$は
重心周りの角運動量を$L_G$、慣性モーメントをそれぞれ$I_{H},I_{S}$として
\begin{eqnarray}
  \omega_{S} = \frac{L_G}{I_{S}} \\
  \omega_{H} = \frac{L_G}{I_{H}}
\end{eqnarray}
と表せる。

離手から倒立までの時間を$t$、初めの胴体の角度を$\theta_0$として、
簡便のために倒立時の姿勢の角度を
\begin{eqnarray}
  \theta_S = \theta_0 + t \omega_S \\
  \theta_H = \theta_0 + t \omega_H
\end{eqnarray}
とする。

回りすぎない、回転不足でない、という条件はそれぞれ
\begin{eqnarray}
  \theta_S > 2 * \pi \\
  \theta_H < 2 * \pi
\end{eqnarray}
と表せば、
\begin{eqnarray*}
  \theta_0 + t \frac{L_G}{I_S} > 2 * \pi \\
  \theta_0 + t \frac{L_G}{I_H} < 2 * \pi
\end{eqnarray*}
\begin{align*}
  \therefore & \quad L_G > \frac{I_S}{t}(2 *\pi-\theta_0)\\
  & \quad L_G < \frac{I_S}{t}(2 *\pi-\theta_0)\\
\end{align*}

\end{document}